\documentclass[11pt,a4paper]{article}

\usepackage[utf8]{inputenc}
\usepackage[T1]{fontenc}
\usepackage{lmodern}
\usepackage{hyperref}
\usepackage{enumitem}

\title{Setting up \LaTeX{} Workshop on Cursor\\
on Rocky Linux 10.1 with TeX Live 2025}
\author{Documented Setup Guide}
\date{\today}

\begin{document}
\maketitle

\tableofcontents

\section{Overview}

This document describes the configuration used to set up a full \LaTeX{}
environment on Rocky Linux 10.1, using:

\begin{itemize}
  \item TeX Live 2025 (scheme-full) installed under \verb|/usr/local/texlive/2025|.
  \item Cursor (VS~Code--like editor).
  \item The \texttt{LaTeX Workshop} extension.
  \item \texttt{latexmk} as the main build tool.
\end{itemize}

The goal is to have:

\begin{itemize}
  \item A complete TeX Live installation to avoid missing packages.
  \item Proper PATH configuration so both the terminal and Cursor can find TeX tools.
  \item LaTeX Workshop configured explicitly to use \texttt{latexmk}.
  \item A simple clean recipe using \texttt{latexmk -c}.
\end{itemize}

\section{Installing TeX Live 2025 (Full Scheme)}

\subsection{Downloading and unpacking the installer}

In a terminal:

\begin{verbatim}
mkdir -p ~/installers
cd ~/installers
wget http://mirror.ctan.org/systems/texlive/tlnet/install-tl-unx.tar.gz
tar xzf install-tl-unx.tar.gz
cd install-tl-*
\end{verbatim}

\subsection{Running the TeX Live installer}

Start the installer with root privileges (for system-wide installation):

\begin{verbatim}
sudo ./install-tl
\end{verbatim}

In the text-based menu:

\begin{enumerate}[label=\arabic*.]
  \item Select the installation scheme:
  \begin{verbatim}
<S> set installation scheme: scheme-full
  \end{verbatim}
  Choose the \textbf{full} scheme (everything) so that almost all
  packages are installed.
  \item Optionally, set letter-size paper as the default (recommended in
  places that use US Letter):
  \begin{verbatim}
<O> options:
  [X] use letter size instead of A4 by default
  \end{verbatim}
  \item Confirm directories (defaults are fine):
  \begin{verbatim}
<D> set directories:
  TEXDIR:        /usr/local/texlive/2025
  TEXMFLOCAL:    /usr/local/texlive/texmf-local
  TEXMFSYSVAR:   /usr/local/texlive/2025/texmf-var
  TEXMFSYSCONFIG:/usr/local/texlive/2025/texmf-config
  TEXMFVAR:      ~/.texlive2025/texmf-var
  TEXMFCONFIG:   ~/.texlive2025/texmf-config
  TEXMFHOME:     ~/texmf
  \end{verbatim}
  \item When satisfied, start the installation:
  \begin{verbatim}
<I> start installation to hard disk
  \end{verbatim}
\end{enumerate}

At the end of the installation, TeX Live prints a message reminding you
to add the \verb|bin|, \verb|MANPATH|, and \verb|INFOPATH| entries to
your environment.

\section{Shell Environment Configuration}

\subsection{Cleaning up install-only environment variables}

If the variable \verb|TEXLIVE_INSTALL_NO_DISKCHECK| was used during the
installation, it is safe---and recommended---to unset it afterwards,
since it is no longer needed:

\begin{verbatim}
unset TEXLIVE_INSTALL_NO_DISKCHECK
\end{verbatim}

Remove any persistent export line for this variable that may have been
added to \verb|~/.bashrc| or \verb|~/.profile|.

\subsection{Adding TeX Live to PATH, MANPATH, and INFOPATH}

Append the following lines to \verb|~/.bashrc|:

\begin{verbatim}
echo 'export PATH=/usr/local/texlive/2025/bin/x86_64-linux:$PATH' \
  >> ~/.bashrc
echo 'export MANPATH=/usr/local/texlive/2025/texmf-dist/doc/man:$MANPATH' \
  >> ~/.bashrc
echo 'export INFOPATH=/usr/local/texlive/2025/texmf-dist/doc/info:$INFOPATH' \
  >> ~/.bashrc
\end{verbatim}

Reload the shell configuration:

\begin{verbatim}
source ~/.bashrc
\end{verbatim}

Then verify that \TeX{} tools are being found:

\begin{verbatim}
which pdflatex
pdflatex --version

which latexmk
latexmk -v

which biber
biber --version
\end{verbatim}

The \verb|which| commands should report paths under
\verb|/usr/local/texlive/2025/bin/x86_64-linux|.

\section{Installing and Configuring Cursor + LaTeX Workshop}

\subsection{Installing the LaTeX Workshop extension}

\begin{enumerate}[label=\arabic*.]
  \item Open Cursor.
  \item Open the Extensions view.
  \item Search for \texttt{LaTeX Workshop} (author: James Yu).
  \item Install the extension.
\end{enumerate}

\subsection{Ensuring PATH inside Cursor}

Graphical applications sometimes do not inherit the same PATH as the
terminal. To ensure that Cursor and the integrated terminal see TeX
Live, we configure the PATH explicitly in the editor settings.

Open the user settings JSON in Cursor:

\begin{enumerate}[label=\arabic*.]
  \item Press \texttt{Ctrl+Shift+P}.
  \item Run \texttt{Preferences: Open User Settings (JSON)}.
\end{enumerate}

Use the following configuration (this is the final, working setup):

\begin{verbatim}
{
  "window.commandCenter": true,

  "terminal.integrated.env.linux": {
    "PATH": "/usr/local/texlive/2025/bin/x86_64-linux:/usr/bin:/bin"
  },

  "latex-workshop.latex.path": "/usr/local/texlive/2025/bin/x86_64-linux",

  // === Tools: define latexmk and clean tools ===
  "latex-workshop.latex.tools": [
    {
      "name": "latexmk",
      "command": "latexmk",
      "args": [
        "-synctex=1",
        "-interaction=nonstopmode",
        "-file-line-error",
        "-pdf",
        "-outdir=%OUTDIR%",
        "%DOC%"
      ]
    },
    {
      "name": "latexmk-clean",
      "command": "latexmk",
      "args": [
        "-c",
        "-outdir=%OUTDIR%",
        "%DOC%"
      ]
    }
  ],

  // === Recipes: define latexmk recipes ===
  "latex-workshop.latex.recipes": [
    {
      "name": "latexmk",
      "tools": ["latexmk"]
    },
    {
      "name": "latexmk-clean",
      "tools": ["latexmk-clean"]
    }
  ],

  "latex-workshop.view.pdf.viewer": "tab"
}
\end{verbatim}

Notes:

\begin{itemize}
  \item The key \verb|"terminal.integrated.env.linux"| ensures the
  integrated terminal has the correct PATH to find TeX Live.
  \item \verb|"latex-workshop.latex.path"| points LaTeX Workshop
  directly to the TeX Live binaries.
  \item A \texttt{latexmk} tool is defined for normal PDF builds.
  \item A \texttt{latexmk-clean} tool is defined to clean auxiliary
  files using \verb|latexmk -c|.
  \item Two recipes are provided: one for building, one for cleaning.
\end{itemize}

After editing \verb|settings.json|, reload the Cursor window
(\emph{Developer: Reload Window}) to ensure the new configuration is
picked up.

\section{Using LaTeX Workshop with \texttt{latexmk}}

\subsection{Testing with a minimal document}

Create a test directory and a simple document:

\begin{verbatim}
mkdir -p ~/tex-test
cd ~/tex-test
\end{verbatim}

Create \verb|main.tex|:

\begin{verbatim}
\documentclass{article}
\usepackage[utf8]{inputenc}
\usepackage{amsmath, amssymb}
\usepackage{hyperref}

\title{TeX Live + LaTeX Workshop Test}
\author{Me}
\date{\today}

\begin{document}
\maketitle

This is a test. Here is an equation:
\begin{equation}
  E = mc^2
\end{equation}

A link: \url{https://example.com}

\end{document}
\end{verbatim}

Then:

\begin{enumerate}[label=\arabic*.]
  \item Open the folder \verb|~/tex-test| in Cursor.
  \item Open \verb|main.tex|.
  \item In the LaTeX Workshop sidebar, select the \texttt{latexmk}
  recipe.
  \item Run ``Build LaTeX project''.
\end{enumerate}

You should see \texttt{latexmk} running in the LaTeX Workshop log, and
the resulting PDF should open in a tab. SyncTeX (jumping between source
and PDF) should also work.

\subsection{Cleaning auxiliary files}

To clean the auxiliary files produced by \texttt{latexmk}, select the
\texttt{latexmk-clean} recipe and run it. This runs:

\begin{verbatim}
latexmk -c -outdir=%OUTDIR% %DOC%
\end{verbatim}

which removes common temporary files but leaves the PDF.

\section{Notes on Bibliography and BibTeX/Biber}

During setup, an error of the following form may appear in the
\texttt{latexmk} log:

\begin{verbatim}
Running 'bibtex  "main.aux"'
This is BibTeX, Version 0.99d (TeX Live 2025)
I found no \citation commands---while reading file main.aux
I found no \bibdata command---while reading file main.aux
I found no \bibstyle command---while reading file main.aux
(There were 3 error messages)
\end{verbatim}

This indicates that \texttt{latexmk} attempted to run BibTeX, but the
document either:

\begin{itemize}
  \item does not contain any bibliography commands at all, or
  \item is not yet properly configured with \verb|\bibliography{...}|
  and \verb|\bibliographystyle{...}| (for classic BibTeX), or with
  BibLaTeX and Biber.
\end{itemize}

For simple test documents with no citations, one can disable automatic
calls to BibTeX by adding the option \verb|-bibtex-| in the
\texttt{latexmk} tool definition, e.g.:

\begin{verbatim}
"args": [
  "-synctex=1",
  "-interaction=nonstopmode",
  "-file-line-error",
  "-pdf",
  "-bibtex-",
  "-outdir=%OUTDIR%",
  "%DOC%"
]
\end{verbatim}

For real projects with bibliographies, ensure that either classic BibTeX
or BibLaTeX+Biber is configured correctly in \verb|main.tex| and that
the corresponding \verb|.bib| files exist.

\section{Summary}

The working configuration consists of:

\begin{itemize}
  \item TeX Live 2025 installed with the full scheme in
  \verb|/usr/local/texlive/2025|.
  \item Environment variables PATH, MANPATH, and INFOPATH extended to
  include the TeX Live directories.
  \item Cursor configured so that its integrated terminal and LaTeX
  Workshop extension can see the TeX Live binaries.
  \item LaTeX Workshop explicitly configured to use \texttt{latexmk}
  for building and \texttt{latexmk -c} for cleaning.
\end{itemize}

With these steps, a full \LaTeX{} project on Rocky Linux 10.1 can be
edited and compiled comfortably in Cursor using LaTeX Workshop.

\end{document}
