\section*{Resumen}

\textbf{Título:} \tit\footnote{Trabajo de Investigación}

\textbf{Autor:} \nam\footnote{\fac.\ \esc.\\ Director: \tdir\ \dir, Codirector: \tcdir\ \cdir }

\textbf{Palabras clave:} { Computación cuántica, Simulación cuántica, Gestión de memoria,\\ Compresión de datos, Computación de Alto Rendimiento (HPC) }

\textbf{Descripción:}

La simulación de sistemas cuánticos en hardware clásico enfrenta una limitación severa debido al crecimiento exponencial en la memoria requerida. A medida que aumenta el número de qubits, la cantidad de datos a almacenar supera rápidamente las capacidades de las computadoras convencionales y supercomputadoras. Esta restricción impide la exploración de algoritmos cuánticos avanzados y dificulta la validación de experimentos en entornos clásicos antes de su ejecución en hardware cuántico. Por ello, es fundamental desarrollar estrategias eficientes de gestión de memoria que permitan extender el alcance de estas simulaciones sin comprometer la fidelidad de los cálculos.
A pesar de los avances en simuladores cuánticos y técnicas de optimización, la gestión de memoria sigue siendo un cuello de botella significativo. En particular, muchas soluciones actuales sacrifican fidelidad a cambio de eficiencia, lo que limita su aplicabilidad en contextos donde la precisión es fundamental. Por ello, es necesario explorar estrategias que reduzcan el consumo de memoria sin comprometer la exactitud de los resultados.
Este trabajo propone el diseño e implementación de una estrategia de compresión de memoria sin pérdida de precisión en simuladores cuánticos, con el objetivo de encontrar un balance óptimo entre uso de memoria, rendimiento y fidelidad. Al final, se espera evaluar la viabilidad de la propuesta y compararla con otros enfoques existentes para determinar su impacto en la optimización de simulaciones cuánticas en hardware clásico.

\newpage

\section*{Abstract}

\textbf{Title:} {Optimization of memory management in quantum simulators through data\\ compression without loss of precision}\footnote{Research Work}

\textbf{Autor:} \nam\footnote{Faculty of Physics-Mechanics Engineering. School of Systems Engineering and Computer
Science.\\ Advisor: \tdir\ \dir, Co-Advisor: \tcdir\ \cdir }

\textbf{Palabras clave:} { Quantum Computing, Quantum Simulation, Memory Management, Data Compression, High-Performance Computing (HPC) }

\textbf{Descripción:}

The simulation of quantum systems on classical hardware faces a severe limitation due to the exponential growth in memory requirements. As the number of qubits increases, the amount of data to be stored quickly exceeds the capabilities of conventional computers and supercomputers. This restriction hinders the exploration of advanced quantum algorithms and complicates the validation of experiments in classical environments before their execution on quantum hardware. Therefore, it is crucial to develop efficient memory management strategies that extend the scope of these simulations without compromising computational fidelity.
Despite advances in quantum simulators and optimization techniques, memory management remains a significant bottleneck. In particular, many current solutions sacrifice fidelity in exchange for efficiency, limiting their applicability in contexts where precision is crucial. Therefore, it is necessary to explore strategies that reduce memory consumption without compromising the accuracy of the results.
This work proposes the design and implementation of a lossless memory compression strategy for quantum simulators, aiming to find an optimal balance between memory usage, performance, and fidelity. Ultimately, the feasibility of the proposal will be evaluated and compared with existing approaches to determine its impact on optimizing quantum simulations on classical hardware.

\newpage