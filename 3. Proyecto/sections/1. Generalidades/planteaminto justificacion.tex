\section{Planteamiento del problema}

La simulación de sistemas cuánticos en computadoras clásicas se enfrenta a una barrera casi insuperable: el crecimiento exponencial de la memoria requerida. Representar un estado cuántico de $n$ qubits implica almacenar $2^{n}$ números complejos, lo que rápidamente supera la capacidad de cualquier sistema de memoria convencional. Las técnicas actuales, como la poda de estados, reducen drásticamente el consumo de recursos pero a costa de la precisión, mientras que la compresión con pérdida, aunque ofrece mayores índices de reducción, sacrifica irremediablemente la fidelidad de los cálculos. Este dilema –la imposibilidad de simular algoritmos cuánticos complejos sin perder información crítica– constituye el núcleo del problema, amenazando la validez y aplicabilidad de las simulaciones en el desarrollo de nuevas estrategias cuánticas.

Este desafío se agrava al considerar que la simulación precisa es esencial para la validación y desarrollo de algoritmos cuánticos en entornos clásicos. La degradación de la precisión debido a la pérdida de datos no solo impide obtener resultados confiables, sino que también limita la escalabilidad y el rendimiento de los simuladores cuánticos. Por ello, es crucial explorar alternativas que permitan optimizar el uso de recursos computacionales sin comprometer la exactitud de los cálculos, destacando la urgente necesidad de investigar métodos de compresión sin pérdida que ofrezcan un equilibrio entre eficiencia y fidelidad en la simulación cuántica.

Con este planteamiento se proponen los siguientes objetivos para abordar esta problemática.